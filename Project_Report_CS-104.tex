\documentclass{article}

\begin{document}

\title{File Organizer Process Report}
\author{Your Name}
\date{\today}
\maketitle

\section{Introduction}
This report documents the process followed to create the Bash script \texttt{organizer.sh}, which is designed to organize files in a directory based on specified criteria. The script was developed on <START_DATE> and completed on <COMPLETION_DATE>.

\section{Process Overview}
The file organizer script was developed in several steps, as outlined below:

\subsection{Step 1: Defining the Problem}
The initial step involved understanding the problem at hand, which was to organize files in a directory based on specific criteria such as file extension or creation date. The goal was to create a flexible and customizable script that could be easily used by users.

\subsection{Step 2: Requirements Gathering}
In this step, the requirements for the script were identified. The key requirements included:
\begin{itemize}
  \item Ability to specify the source and destination directories
  \item Option to choose the organization style (extension-based or date-based)
  \item Ability to delete original files after organizing
  \item Option to exclude specific file types or directories
  \item Generating a log file with the details of the organizing process
  \item Compressing specific files into tarballs
\end{itemize}

\subsection{Step 3: Script Design}
Based on the requirements, the script's design was conceptualized. The script would be written in Bash, a widely used scripting language for Unix-like operating systems. The script would utilize functions to handle different aspects of the file organizing process, including folder creation, file movement, and file compression.

\subsection{Step 4: Script Implementation}
The script was implemented according to the designed specifications. The Bash scripting language provided the necessary constructs to handle file operations, such as finding files, creating folders, moving files, and generating log files. The script utilized loops, conditionals, and command-line arguments to achieve the desired functionality.

\subsection{Step 5: Testing and Refinement}
Once the script was implemented, it was thoroughly tested to ensure its correctness and reliability. Test cases were created to cover various scenarios, including different organization styles, excluded file types, and file compression. Any issues or bugs identified during testing were addressed, and the script was refined to improve its performance and usability.

\section{Conclusion}
The file organizer script (\texttt{organizer.sh}) has been successfully developed to organize files based on specific criteria. The script provides a flexible and customizable solution for organizing files in a directory. It allows users to choose the organization style, delete original files if desired, exclude specific file types or directories, generate a log file, and compress selected files into tarballs.

The development of the script involved a systematic process, including problem understanding, requirements gathering, script design, implementation, and testing. The final script meets the specified requirements and has undergone rigorous testing to ensure its functionality.

\end{document}
